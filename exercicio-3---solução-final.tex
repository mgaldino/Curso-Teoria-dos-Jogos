% Options for packages loaded elsewhere
\PassOptionsToPackage{unicode}{hyperref}
\PassOptionsToPackage{hyphens}{url}
%
\documentclass[
]{article}
\usepackage{lmodern}
\usepackage{amssymb,amsmath}
\usepackage{ifxetex,ifluatex}
\ifnum 0\ifxetex 1\fi\ifluatex 1\fi=0 % if pdftex
  \usepackage[T1]{fontenc}
  \usepackage[utf8]{inputenc}
  \usepackage{textcomp} % provide euro and other symbols
\else % if luatex or xetex
  \usepackage{unicode-math}
  \defaultfontfeatures{Scale=MatchLowercase}
  \defaultfontfeatures[\rmfamily]{Ligatures=TeX,Scale=1}
\fi
% Use upquote if available, for straight quotes in verbatim environments
\IfFileExists{upquote.sty}{\usepackage{upquote}}{}
\IfFileExists{microtype.sty}{% use microtype if available
  \usepackage[]{microtype}
  \UseMicrotypeSet[protrusion]{basicmath} % disable protrusion for tt fonts
}{}
\makeatletter
\@ifundefined{KOMAClassName}{% if non-KOMA class
  \IfFileExists{parskip.sty}{%
    \usepackage{parskip}
  }{% else
    \setlength{\parindent}{0pt}
    \setlength{\parskip}{6pt plus 2pt minus 1pt}}
}{% if KOMA class
  \KOMAoptions{parskip=half}}
\makeatother
\usepackage{xcolor}
\IfFileExists{xurl.sty}{\usepackage{xurl}}{} % add URL line breaks if available
\IfFileExists{bookmark.sty}{\usepackage{bookmark}}{\usepackage{hyperref}}
\hypersetup{
  pdftitle={Lista de exercício 3 - solução final},
  pdfauthor={Manoel Galdino},
  hidelinks,
  pdfcreator={LaTeX via pandoc}}
\urlstyle{same} % disable monospaced font for URLs
\usepackage[margin=1in]{geometry}
\usepackage{longtable,booktabs}
% Correct order of tables after \paragraph or \subparagraph
\usepackage{etoolbox}
\makeatletter
\patchcmd\longtable{\par}{\if@noskipsec\mbox{}\fi\par}{}{}
\makeatother
% Allow footnotes in longtable head/foot
\IfFileExists{footnotehyper.sty}{\usepackage{footnotehyper}}{\usepackage{footnote}}
\makesavenoteenv{longtable}
\usepackage{graphicx,grffile}
\makeatletter
\def\maxwidth{\ifdim\Gin@nat@width>\linewidth\linewidth\else\Gin@nat@width\fi}
\def\maxheight{\ifdim\Gin@nat@height>\textheight\textheight\else\Gin@nat@height\fi}
\makeatother
% Scale images if necessary, so that they will not overflow the page
% margins by default, and it is still possible to overwrite the defaults
% using explicit options in \includegraphics[width, height, ...]{}
\setkeys{Gin}{width=\maxwidth,height=\maxheight,keepaspectratio}
% Set default figure placement to htbp
\makeatletter
\def\fps@figure{htbp}
\makeatother
\setlength{\emergencystretch}{3em} % prevent overfull lines
\providecommand{\tightlist}{%
  \setlength{\itemsep}{0pt}\setlength{\parskip}{0pt}}
\setcounter{secnumdepth}{-\maxdimen} % remove section numbering

\title{Lista de exercício 3 - solução final}
\author{Manoel Galdino}
\date{2023-03-24}

\begin{document}
\maketitle

\begin{enumerate}
\def\labelenumi{\arabic{enumi}.}
\tightlist
\item
  Escreva as informações que especificam um jogo na forma normal (Número
  de jogadores, estratégias e funções de utilidade) e a matriz de payoff
  dos seguintes jogos apresentados em aula: Dilema do Prisioneiro, Bach
  e Stravinsky, Jogo do Chicken e Stag Hunt.
\end{enumerate}

Resposta. 1.1 DP: Designando confessar por C e não confessar por NC,
temos: o DP é formado por: jogadores \(N = \{1,2\}\), estratégias
\(S_i = {C, NC}\), e payoffs \(U_i(s_1=C, s_2=C) = -2, i = \{1,2\}\),
\(U_i(s_1=NC, s_2=NC) = -5, i = \{1,2\}\), \(U_1(s_1=C, s_2=NC) = 0\) e
\(U_2(s_1=C, s_2=NC) = -10\), \(U_1(s_1=NC, s_2=C) = -10\) e
\(U_2(s_1=NC, s_2=C) = 0\).

A matriz de payoff é:

\begin{longtable}[]{@{}lll@{}}
\toprule
& C & NC\tabularnewline
\midrule
\endhead
C & (-2,-2) & (-10,0)\tabularnewline
NC & (0,-10) & (-5,-5)\tabularnewline
\bottomrule
\end{longtable}

ps.: se você deixar claro que se trata de anos de prisão na matrriz de
payoff, os números podem ser positivos. Porém, a função utilidade tem de
ter número negativo no DP, pois ela retorna um número real (que mede
utilidade), e não anos de prisão.

1.2

Bach e Stravinsky. Designando Bach por B e Stravinsky por S, temos:

jogadores \(N = \{1,2\}\), estratégias \(S_i = {B, S}\), e payoffs
\(U_1(s_1=B, s_2=B) = 2\) e \(U_2(s_1=B, s_2=B) = 1\),
\(U_i(s_1=B, s_2=S) = 0, i = \{1,2\}\),
\(U_i(s_1=S, s_2=B) = 0, i = \{1,2\}\), \(U_1(s_1=S, s_2=S) = 1\) e
\(U_2(s_1=S, s_2=S) = 2\)

A matriz de payoff é:

\begin{longtable}[]{@{}lll@{}}
\toprule
& B & S\tabularnewline
\midrule
\endhead
B & (2,1) & (0,0)\tabularnewline
S & (0,0) & (1,2)\tabularnewline
\bottomrule
\end{longtable}

1.3 Chicken

Designando Desvia por D e Não desvia por ND, temos:

jogadores \(N = \{1,2\}\), estratégias \(S_i = {, S}\), e payoffs
\(U_1(s_1=B, s_2=B) = 2\) e \(U_2(s_1=B, s_2=B) = 1\),
\(U_i(s_1=B, s_2=S) = 0, i = \{1,2\}\),
\(U_i(s_1=S, s_2=B) = 0, i = \{1,2\}\), \(U_1(s_1=S, s_2=S) = 1\) e
\(U_2(s_1=S, s_2=S) = 2\)

\begin{longtable}[]{@{}lll@{}}
\toprule
& D & ND\tabularnewline
\midrule
\endhead
D & (0,0) & (-1,2)\tabularnewline
ND & (2,-1) & (-10,-10)\tabularnewline
\bottomrule
\end{longtable}

\begin{enumerate}
\def\labelenumi{\arabic{enumi}.}
\setcounter{enumi}{1}
\tightlist
\item
  Para os jogos da questão anterior, responda às seguintes perguntas:
\end{enumerate}

\begin{enumerate}
\def\labelenumi{\alph{enumi})}
\tightlist
\item
  Existe alguma estratégia estritamente dominada para algum (ou ambos)
  jogador(es)? Se sim, qual.
\item
  existe um equilírio de estratégia estritamente dominante? Se sim,
  qual?
\item
  Existe algum equilíbrio de Eliminação Iterativa de Estratégias
  Estritamente Dominadas? Se sim, qual? Ele é ótimo de Pareto?
\end{enumerate}

\begin{enumerate}
\def\labelenumi{\arabic{enumi}.}
\setcounter{enumi}{5}
\tightlist
\item
  Em um pênalti no jogo de futebol, suponha que os jogadores podem
  chutar na esquerda, no centro ou no canto direito (desconsidere a
  altura da bola). Similarmente, o goleiro pode escolher umas das três
  opções para tentar agarrar a bola. Suponha também que batedores e
  goleiros devem escolher simultaneamente aonde a bola irá. Se o goleiro
  pular na mesma direção da bola, o goleiro ganha e o batedor perde. Se
  pular em uma direção diferente, o batedor ganha. Modele esse jogo como
  um jogo na forma normal (isto é, escreva o número de jogadores,
  estratégias e funções de utilidade dos jogadores) e a matriz de
  payoff.
\end{enumerate}

Resposta (matriz de payoff):

Uma vez que os payoffs podem ser números arbitrários, desde que
representem a ordem, iremos considerar que o goleiro defender o pênalti
dá payoff 1 para ele, e zero para o jogador e que se for gol, é o
contrário (1 para o jogador, zero para o goleiro). Assim, colocando o
goleiro nas linhas e o jogador nas colunas, a matriz de payoff fica:

\begin{longtable}[]{@{}llll@{}}
\toprule
& Esquerda & Centro & Direita\tabularnewline
\midrule
\endhead
Esquerda & (1,0) & (0,1) & (0,1)\tabularnewline
Centro & (0,1) & (1,0) & (0,1)\tabularnewline
Direita & (0,1) & (0,1) & (1,0)\tabularnewline
\bottomrule
\end{longtable}

\begin{enumerate}
\def\labelenumi{\arabic{enumi}.}
\setcounter{enumi}{6}
\tightlist
\item
  Três jogadoras vivem em um bairro e podem contribuir para custear uma
  iluminação de um poste. O valor da iluminação é \(3\) unidades
  monetárias para cada jogadora e o valor de ficarem sem iluminação é
  \(0\). A associação do Bairro pede que cada jogadora contribua \(1\)
  unidade monetária cada (ou nada). Se pelo menos duas jogadoras
  contribuírem, a iluminação é instalada. Se uma ou nenhuma pessoa
  contribuir, a iluminação não é instalada e quem deu o dinheiro não
  terá ele de volta. Escreva a forma normal do jogo.
\end{enumerate}

Resposta (matriz de payoff):

Podemos escrever duas matrizes de payoff. A primeira considerando quando
a terceira jogadora contribui e a segunda matriz quando a terceira
jogadora não contribui. Então fica assim:

Primeria matriz de payoff:

\begin{longtable}[]{@{}lll@{}}
\toprule
& Contribui & Não contribui\tabularnewline
\midrule
\endhead
Contribui & (2,2,2) & (2,3,2)\tabularnewline
Não contribui & (3,2,2) & (0,0,-1)\tabularnewline
\bottomrule
\end{longtable}

Segunda matriz de payoff

\begin{longtable}[]{@{}lll@{}}
\toprule
& Contribui & Não contribui\tabularnewline
\midrule
\endhead
Contribui & (2,2,3) & (-1,0,0)\tabularnewline
Não contribui & (0,-1,0) & (0,0,0)\tabularnewline
\bottomrule
\end{longtable}

\begin{enumerate}
\def\labelenumi{\arabic{enumi}.}
\setcounter{enumi}{7}
\tightlist
\item
  Em uma rede social, suponha que temos apenas três pessoas que se
  seguem (e não podem bloquear umas às outras) e podem fazer uma
  postagem sensacionalista ou não. Uma postagem sensacionalista atrai
  mais engajamento e mais seguidores do que a não-sensacionalista. Elas
  gostam de ganhar mais seguidores e não se importam se sua postagem é
  sensacionalista. Porém, se elas vêem em sua linha do tempo postagem
  sensacionalista de outras pessoas, acham isso ruim. Assim, suas
  preferências podem ser representadas pelos seguintes payoffs. Se uma
  jogadora posta algo sensacionalista, ganha \(3\) e se posta algo não
  sensacionalista, ganha \(1\). Se ela nao vir nenhuma mensagem
  sensancionalista alheia em sua linha do tempo, nem ganha nem perde
  nada (ganha zero). Se ela vir uma postagem alheia, perde 2 e se vir
  duas postagens alheias, perde 4. Todas têm as mesmas preferências.
  Suponha que devem decidir simultaneamente se irão postar algo
  sensacionalista ou não (rodada única).
\end{enumerate}

Resposta (matriz de pyaoff): Os payoffs são da seguinte maneira. 3
postagens sensacionalistas geram payoff de -1 para todas. 2 postagens
sensacionalistas geram payoff de -3 para quem faz postagem não
sensacionalista, e 1 para quem faz postagem sensacionalista. 1 post
sensacionalista gera 3 para quem faz o post e -1 para quem não faz post
sensacionalista. Por fim, payoff de 1 se todas postam mensagens
não-sensacionalistas.

As estratégias são postagem sensacionalista ou não sensacionalista. Como
no item anterior, vamos considerar primeiro que a jogadora 3 faz
postagem sensacionalista, e depois que faz não sensacionalista.

Primeria matriz de payoff:

\begin{longtable}[]{@{}lll@{}}
\toprule
& Sensacionalista & Não sensacionalista\tabularnewline
\midrule
\endhead
Sensacionalista & (-1,-1,-1) & (1,-3,1)\tabularnewline
Não sensacionalista & (-3,1,1) & (-1,-1,3)\tabularnewline
\bottomrule
\end{longtable}

Segunda matriz de payoff

\begin{longtable}[]{@{}lll@{}}
\toprule
& Sensacionalista & Não sensacionalista\tabularnewline
\midrule
\endhead
Sensacionalista & (1,1,-3) & (3,-1,-1)\tabularnewline
Não sensacionalista & (-1,3,-1) & (1,1,1)\tabularnewline
\bottomrule
\end{longtable}

\begin{enumerate}
\def\labelenumi{\arabic{enumi}.}
\setcounter{enumi}{8}
\tightlist
\item
  Há dois bares em uma cidade, cujas proprietárias são A e B, que podem
  cobrar 2 reais, 4 reais ou 5 reais por bebida. Todos os dias, há 6.000
  turistas e 4.000 moradores locais que decidem qual bar visitar. (Cada
  pessoa pode ir apenas a um bar e cada pessoa deve ir a pelo menos um
  bar, onde cada pessoa consome exatamente uma bebida.) Como os turistas
  não têm ideia sobre os bares, eles escolhem aleatoriamente sem levar
  em consideração os preços. No entanto, os moradores locais sempre vão
  ao bar mais barato (e escolhem aleatoriamente se os preços forem os
  mesmos).
\end{enumerate}

Resposta (matriz de pyaoff):

\begin{longtable}[]{@{}llll@{}}
\toprule
& 2 & 4 & 5\tabularnewline
\midrule
\endhead
2 & (10,10) & (14,12) & (14,15)\tabularnewline
4 & (12,14) & (20,20) & (28,15)\tabularnewline
5 & (15,14) & (15,28) & (25,25)\tabularnewline
\bottomrule
\end{longtable}

\begin{enumerate}
\def\labelenumi{\arabic{enumi}.}
\setcounter{enumi}{9}
\tightlist
\item
  Considere o seguinte cenário de leilão. Duas pessoas, jogadora 1 e
  jogadora 2, estão competindo para obter um objeto de valor. Cada
  jogadora faz um lance em um envelope lacrado sem saber o lance da
  outra jogadora. Os lances devem ser em múltiplos de 100 reais e o
  valor máximo do lance é de 500 reais. O objeto tem um valor de 400
  reais para a jogadora 1 e 300 reais para a jogadora 2. A licitante com
  o lance mais alto ganha o objeto. Em caso de empate, a jogadora 1 fica
  com o objeto. O vencedor do objeto paga o valor do seu lance. Se ela
  não ganhar o objeto, seu payoff é zero.
\end{enumerate}

Resposta (matriz de pyaoff):

\begin{longtable}[]{@{}lllllll@{}}
\toprule
& 0 & 100 & 200 & 300 & 400 & 500\tabularnewline
\midrule
\endhead
0 & (400,0) & (0,200) & (0,100) & (0,0) & (0,-100) &
(0,-200)\tabularnewline
100 & (300,0) & (300,0) & (0,100) & (0,0) & (0,-100) &
(0,-200)\tabularnewline
200 & (200,0) & (200,0) & (200,0) & (0,0) & (0,-100) &
(0,-200)\tabularnewline
300 & (100,0) & (100,0) & (100,0) & (100,0) & (0,-100) &
(0,-200)\tabularnewline
400 & (0,0) & (0,0) & (0,0) & (0,0) & (0,0) & (0,-200)\tabularnewline
500 & (-100,0) & (-100,0) & (-100,0) & (-100,0) & (-100,0) &
(-100,0)\tabularnewline
\bottomrule
\end{longtable}

\end{document}
