\documentclass{article}

\usepackage{tikz}
\usetikzlibrary{positioning}

\usepackage{Sweave}
\begin{document}
\Sconcordance{concordance:jogo-confianca.tex:jogo-confianca.Rnw:1 5 1 1 0 30 1}


\begin{tikzpicture}[>=stealth, every node/.style={circle}]
\tikzstyle{level 1}=[sibling distance=50mm, level distance=25mm]
\tikzstyle{level 2}=[sibling distance=25mm]

% Estilos dos nós
\tikzstyle{solid node}=[draw, circle, fill=black, inner sep=1.5, minimum size=2mm]

% Definindo os estilos para as linhas dos jogadores
\tikzstyle{player 1}=[solid node, label=above:{1}]
\tikzstyle{player 2}=[solid node, label=above:{2}]

% Jogo da Confiança
\node[player 1] (x_0) {}
  child {node[label=center:{$(0,0)$}](x_1) {} edge from parent node[left] {Não}}
  child {node[player 2] (x_2) {}
    child {node[label=center:{$(1,1)$}] (x_3) {} edge from parent node[left] {Cooperar}}
    child {node[label=center:{$(-1,2)$}] (x_4) {} edge from parent node[right] {Não cooperar}}
    edge from parent node[right] {Confiar}
  };
  
% Adicionando rótulo de numeração manualmente para x_0
\node at (x_0) [yshift=-4mm] {$x_{0}$};
\node at (x_2) [yshift=-4mm] {$x_{2}$};
% Adicionando rótulos de numeração para o restante usando foreach
\foreach \x in {1,3,4} 
  \node at (x_\x) [yshift=4mm] {$x_{\x}$};
\end{tikzpicture}
\end{document}
